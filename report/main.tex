\documentclass{article}

\usepackage[left=10mm,right=10mm,top=10mm,bottom=20mm,paper=a4paper]{geometry}

% межстрочный интервал
\linespread{1.5}

% top page pagination
\usepackage{fancyhdr}
\fancyhf{}
\fancyhead[C]{\thepage}
\renewcommand\headrulewidth{0pt}

% times new roman support
\usepackage{fontspec}
\setmainfont{Times New Roman}

% more font sizes
\usepackage{scrextend}

% Formatting TOC
\usepackage{tocloft}
\renewcommand{\cftsecfont}{\normalfont\mdseries}% titles in non-bold
\renewcommand{\cftsecpagefont}{\normalfont\mdseries}% page numbers in non-bold
\renewcommand{\cftsecleader}{\mdseries\cftdotfill{\cftdotsep}}% dot leaders in non-bold
\renewcommand{\contentsname}{}
\renewcommand{\cftsecaftersnum}{.}
\renewcommand{\cftsubsecaftersnum}{.}
\usepackage{titlesec}
\titleformat{\section}{\normalfont\center\bfseries\uppercase}{}{0em}{}
\titleformat{\subsection}{\normalfont\center\bfseries}{\thesubsection}{0.5em}{}

% remove footnote margin
\usepackage{footmisc}
\usepackage{url}
\urlstyle{same}

% tables support
\usepackage{float}
\usepackage{multirow}
\usepackage{tabularx}
\usepackage{tablefootnote}
\usepackage{caption} 
% custom captions
\renewcommand{\figurename}{Рис.}
\renewcommand{\tablename}{Таблица}
\captionsetup[table]{labelsep=endash,justification=centering,singlelinecheck=false,font=normalsize}

% image support
\usepackage{graphicx}
\graphicspath{ {images/} }

% math symbols (R, C, N)
\usepackage{amssymb}


\title{lab}
\author{}
\date{}

\begin{document}

% абзацный отступ
\setlength{\parindent}{35pt}
% remove footnote margin
\setlength{\footnotemargin}{5pt}
% text justification
\sloppy
\frenchspacing % fixes double spaces after commas

\KOMAoption{fontsize}{14pt}


\begin{table}[!h]
    \begin{center}
        \begin{tabular}{ | m{4.2em} | m{5em} | m{5em} |}
            \hline
            Бенчмарк & Время & Расстояние \\
            \hline
            tai20a   & 0.012 & 730030     \\
            \hline
            tai40a   & 0.178 & 3276438    \\
            \hline
            tai60a   & 0.807 & 7496014    \\
            \hline
            tai80a   & 2.839 & 13953580   \\
            \hline
            tai100a  & 7.986 & 21745312   \\
            \hline
        \end{tabular}
        \caption{Результаты бенчмарков алгоритма локального поиска с лучшим улучшением. Время в секундах.}
    \end{center}
\end{table}

\begin{table}[!h]
    \begin{center}
        \begin{tabular}{ | m{4.2em} | m{5em} | m{5em} |}
            \hline
            Бенчмарк & Время    & Расстояние \\
            \hline
            tai20a   & 1.395    & 721950     \\
            \hline
            tai40a   & 23.686   & 3223474    \\
            \hline
            tai60a   & 127.475  & 7464750    \\
            \hline
            tai80a   & 420.46   & 13918096   \\
            \hline
            tai100a  & 1034.976 & 21680094   \\
            \hline
        \end{tabular}
        \caption{Результаты бенчмарков алгоритма итерационного локального поиска с лучшим улучшением. Время в секундах.}
    \end{center}
\end{table}

Результаты времени работы алгоритмов получены на основе 10 запусков.
В итерационном алгоритме было 10 итераций.

Ответы каждого алгоритма на каждую задачу представлены в соостветствющих файлах в
answers / <\textit{algo\_name}> / <\textit{benchmark\_name}>.sol.

Целевая функция и быстрый пересчет свопа двух элементов были реализованы в соостветствии с формулами с лекции.

\end{document}
